\documentclass{article}
\usepackage{tikz}
\usepackage{pgfplots}
\usepackage{amsmath}
\usepackage{amssymb}
\begin{document}
\title{ASSIGNMENT 4}
\author{Showqeen Yousuf}
\date{\today}
\maketitle

\begin{itemize}

\item{\textbf{Exercise:2.19:}}\\

Find the equation of the line parallel to the line \[ \begin{pmatrix} 3 & -4\end{pmatrix}x =2\] and passing through the point\[ \begin{pmatrix} -2\\ 4\end{pmatrix}\]\\

\item{\textbf{Solution:}}\\

Given,\\
a line to which our line is parallel,
\[ \begin{pmatrix} 3 & -4\end{pmatrix}x =2\]\\
and  the point \[ \begin{pmatrix} -2\\ 4\end{pmatrix}\]\\ via which our line passes.\\

Here, slope of our line and of given line will be equal or same.\\

\therefore{}& $m_1=m$.\\

Here,
\[ n_1=\begin{pmatrix} 3 \\ -4\end{pmatrix}\] and\\
\[m_1= \begin{pmatrix} 4 \\ 3\end{pmatrix}\]\\

$Slope_1=3/4$\\

For the required line $slope=slope_1=3/4$\\

\[m=m_1= \begin{pmatrix} 4 \\ 3\end{pmatrix}\] and\\
\[ n=n_1=\begin{pmatrix} 3 \\ -4\end{pmatrix}\]\\

\therefore{}& $equation of required line is$\\

\hspace{4.5 cm}$X=A+ {\lambda}_1*m$\\

\[ X=\begin{pmatrix} -2 \\ 3\end{pmatrix}+ {\lambda}_1\begin{pmatrix} 4 \\ 3\end{pmatrix}\]\\

Or equation of required line is,\\

\hspace{4.5 cm}$n^{T}(X-A)=0$\\
\[ \begin{pmatrix} 3 & -4\end{pmatrix}(X-\begin{pmatrix} -2 \\ 3\end{pmatrix}=0\]\\
\[ \begin{pmatrix} 3 & -4\end{pmatrix}(X)+6+12=0\]\\
\[ \begin{pmatrix} 3 & -4\end{pmatrix}(X)=-18\]
\newpage
\item{\textbf{Exercise:2.21:}}\\

The hypotenuse of a right angled triangle has its ends at the points \[ \begin{pmatrix} 1 \\ 3\end{pmatrix}\] and \[ \begin{pmatrix} -4\\ 1\end{pmatrix}\]\\Find an equation of the legs of the triangle.

\item{\textbf{Solution:}}\\

Given,\\
the end points  of hypotenuse line \[ \begin{pmatrix} 1 \\ 3\end{pmatrix}\] and \[ \begin{pmatrix} -4\\ 1\end{pmatrix}\]
Assuming a particular condition, then the other vertex of the right angeled triangle will be\[ \begin{pmatrix} 1 \\ 1\end{pmatrix}\]
\begin{center}
\begin{tikzpicture}
\draw(-4,1)--(1,3)--(1,1)--cycle;
\node at(1,3)[above left]{$C$};
\node at(-4,1)[below left]{$A$};
\node at(1,1)[below right]{$B$};
\end{tikzpicture}
\end{center}

$
 
i) For equation of AB(base)\\

\hspace{4,5 cm}m=B-A\\
\[m=\begin{pmatrix} 1 \\ 1\end{pmatrix}-\begin{pmatrix} -4 \\ 1\end{pmatrix}=\begin{pmatrix} 4 \\ 0\end{pmatrix}\]\\

\[n=\begin{pmatrix} 0 \\ 4\end{pmatrix}\]
For equation,\\

\hspace{4,5 cm}$n^{T}(X-A)=0$\\
\[ \begin{pmatrix} 0 & 4\end{pmatrix}(X-\begin{pmatrix} 1 \\ 1\end{pmatrix}=0\]\\
\[ \begin{pmatrix} 0 & 4\end{pmatrix}(X)-4=0\]\\
\[ \begin{pmatrix} 0 & 4\end{pmatrix}(X)=4\]

ii) For equation of BC(altitude)\\

\hspace{4,5 cm}m=C-B\\
\[m=\begin{pmatrix} 1 \\ 3\end{pmatrix}-\begin{pmatrix} 1 \\ 1\end{pmatrix}=\begin{pmatrix} 0 \\ 2\end{pmatrix}\]\\

\[n=\begin{pmatrix} -2 \\ 0\end{pmatrix}\]
For equation,\\

\hspace{4,5 cm}$n^{T}(X-B)=0$\\
\[ \begin{pmatrix} -2 & 0\end{pmatrix}(X-\begin{pmatrix} 1 \\ 1\end{pmatrix}=0\]\\
\[ \begin{pmatrix} -2 & 0\end{pmatrix}(X)+2=0\]\\
\[ \begin{pmatrix} -2 & 0\end{pmatrix}(X)=-2\]
\end{itemize}
\end{document}